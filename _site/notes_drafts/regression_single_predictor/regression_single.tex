\documentclass[10pt]{amsart}
%\usepackage{amsmath}
\usepackage[utf8]{inputenc}
\usepackage{graphicx}
\usepackage{xcolor}
\usepackage[legalpaper, margin=1.5in]{geometry}
\usepackage{fancyvrb}
\usepackage{url}
\usepackage{etoolbox}
\usepackage{hyperref}




% fonts
\input{ArtNouvc.fd}
\newcommand*\initfamily{\usefont{U}{ArtNouvc}{xl}{n}}
\usepackage[T1]{fontenc}
\usepackage{tgbonum} % change font


% from https://www.pinterest.co.uk/pin/88664686405122253/
\definecolor{C1}{RGB}{141, 125, 158} %purple
\definecolor{C2}{RGB}{163,156,147} % yellow
\definecolor{C3}{RGB}{99,151,153} % green
\definecolor{C4}{RGB}{195,106,99} % red
\definecolor{C5}{RGB}{124, 132, 128}
\definecolor{C6}{RGB}{67, 69, 75}

\hypersetup{
    colorlinks=false,
    linkcolor=C2,
    filecolor=magenta,      
    urlcolor=C3,
    pdftitle={Overleaf Example},
    pdfpagemode=FullScreen,
    }
    
    
\patchcmd{\section}{\normalfont}{\color{C5}}{}{}
\patchcmd{\subsection}{\normalfont}{\color{C5}}{}{}


\newcommand{\dfn}[1]{\bf  \color{C2}{#1}}


\renewcommand{\FancyVerbFormatLine}[1]{\color{C3}{>\,\,#1}}
    
\newtheorem{thm}{Theorem}

\newtheoremstyle{exercise}
{}                % Space above
{}                % Space below
{\color{C4}\slshape}        % Theorem body font % (default is "\upshape")
{}                % Indent amount
{\bfseries}       % Theorem head font % (default is \mdseries)
{:}               % Punctuation after theorem head % default: no punctuation
{ }               % Space after theorem head
{}                % Theorem head spec
\theoremstyle{exercise}
\newtheorem{exercise}{Exercise}


\newtheoremstyle{example}
{}                % Space above
{}                % Space below
{\color{C1}\slshape}        % Theorem body font % (default is "\upshape")
{}                % Indent amount
{\bfseries}       % Theorem head font % (default is \mdseries)
{.}               % Punctuation after theorem head % default: no punctuation
{ }               % Space after theorem head
{}                % Theorem head spec
\theoremstyle{example}
\newtheorem{example}{Example}

\newtheoremstyle{solution}
{}                % Space above
{}                % Space below
{\color{C5}}        % Theorem body font % (default is "\upshape")
{}                % Indent amount
{\bfseries }       % Theorem head font % (default is \mdseries)
{:}               % Punctuation after theorem head % default: no punctuation
{\newline}               % Space after theorem head
{}                % Theorem head spec
\theoremstyle{solution}
\newtheorem*{solution}{Solution}


\newcommand{\mcS}{\mathcal S}
\newcommand{\mcR}{\mathcal R}
\newcommand{\mcC}{\mathcal C}
\newcommand{\bS}{{\boldsymbol S}}
\newcommand{\bR}{{\boldsymbol R}}
\newcommand{\bC}{{\boldsymbol C}}
\newcommand{\ba}{{\boldsymbol a}}
\newcommand{\bb}{{\boldsymbol b}}
\newcommand{\bs}{{\boldsymbol s}}
\newcommand{\bff}{{\boldsymbol f}}
\newcommand{\br}{{\boldsymbol r}}
\newcommand{\bx}{{\boldsymbol x}}
\newcommand{\bt}{{\boldsymbol t}}
\newcommand{\bv}{{\boldsymbol v}}
\newcommand{\bu}{{\boldsymbol u}}
\newcommand{\bw}{{\boldsymbol w}}
\newcommand{\bc}{{\boldsymbol c}}
\newcommand{\be}{{\boldsymbol e}}
\newcommand{\bq}{{\boldsymbol q}}
\newcommand{\bphi}{{\boldsymbol \phi}}
\newcommand{\brho}{{\boldsymbol \rho}}
\newcommand{\btau}{{\boldsymbol \tau}}
\newcommand{\reals}{\mathbb R}
\newcommand{\ints}{\mathbb N}
\newcommand{\E}{\mathbb E}
\newcommand{\Prob}{\mathbb P}

\title{\Huge \color{C1}  Linear regression}
\author{Ethan Levien}
\date{April 2022}

\begin{document}

\maketitle

\tableofcontents




\section{The linear regression model with one predictor}
The main subject of this course is linear regression models. In the simplest case where we have a single independent variable, $x$, a regression model for the relationship between $Y$ and  $x$ is 
\begin{equation}\label{eq:reg}
y \sim {\rm Normal}(a+ b x,\sigma)
\end{equation}
The variable $x$ is called the {\bf predictor}.  Written another way, our model for $Y$ is 
\begin{equation}
y =a+ bx + \xi,\quad \xi \sim {\rm Normal}(0,\sigma).
\end{equation}


Importantly, the standard linear regression model is a conditional model, in the sense that is tell us the distribution of our dependent variable, $y$, \emph{conditioned} on $x$. This means that the distribution of $x$ is not part of our model and in order to make predictions, the values of $x$ need to be given to us. In order to emphasize this point, we sometimes write 
\begin{equation}\label{eq:reg}
y|x \sim {\rm Normal}(a+ b x,\sigma)
\end{equation}

We are interested in the associated statistical inference problem, but first, let's assume $a$ and $b$ are known and think about some of the predictions we can make.  

\begin{example} Suppose that we model the temperature $y$ in degrees Fahrenheit on a given day as a function of the temperature on a previous day 
\begin{equation}
y \sim {\rm Normal}(0.5x+ 30,3.)
\end{equation}
If the temperature was $88$ yesterday, what is the chance it is greater than $88$ tomorrow? Use properties of the Normal distribution and confirm your result with simulations
\end{example}
\begin{solution}
$y$ is Normal with a mean and standard deviation of $70$ and $3$ respectively. From properties of the Normal distribution, the chance that $y>$ is $64$. 
\begin{Verbatim}
y = np.random.normal(70,5,1000)
len(y[y>70])/len(y)
\end{Verbatim}
\end{solution}



\begin{example} Suppose that we model the temperature $y$ in degrees Fahrenheit on a given day as a function of the temperature on a previous day 
\begin{equation}
y \sim {\rm Normal}(0.5x+ 30,1.)
\end{equation}
If the temperature was $88$ yesterday, what is the chance it is greater than $88$ tomorrow? 
\end{example}

\begin{exercise} Continuing with the temperature model from the previous question,  
\end{exercise}



\url{https://colab.research.google.com/drive/1H24olTdEAPp2xpxhMXhyvFaEzjjaiZnk#scrollTo=HdT4XpUZoVHk&line=1&uniqifier=1}

\section{Inference for linear regression models}
Now suppose we have some data $D$ consisting of $x$ and $y$ pairs:
\begin{equation}
D = \{(x_1,y_1),\dots,(x_n,y_n)\}
\end{equation}


One can show that the MLE estimates of $\beta$ and $\alpha$ are 
\begin{align}
\hat{b}_{\rm MLE} &= \frac{\sum (x_i - \bar{x})(y_i - \bar{y})}{\sum (x_i-\bar{x})^2} \\
\hat{a}_{\rm MLE} &= \bar{y} - \hat{\beta} \bar{x}
\end{align}
where $\bar{x}$ and $\bar{y}$ are the sample means of $x$ and $y$ respectively. 
This approach to estimating the regression slopes is also known as {\bf least squares}, because it minimizes the squared error between the {\bf residuals}
 \begin{equation}
r_i =\hat{b} x_i + \hat{a}- y_i. 
\end{equation}


\begin{exercise}  Discuss the relationship between the residuals and the noise term $\xi_i$, the difference between the data and the regression line.   Are they equal? Why or why not? 
\end{exercise}



There is another parameter in our model, $\hat{\sigma}$. This also has an MLE.  Recall that the estimator of a the standard deviation of a Normal distribution is 
\begin{equation}\label{eq:normal_std}
\sqrt{\frac{1}{n-1}}\sum y_i^2
\end{equation}
If we want to estimate the standard error, we might expect to replace $y_i$ with $r_i$ (can you see why?); however, this does not account for the fact that we don't know $\hat{\alpha}_i$. This additional degree of freedom causes the Equation \eqref{eq:normal_std} to over estimate the variance. Correcting for this yields  
\begin{equation}
\hat{\sigma} = \sqrt{\frac{1}{n-2}}\sum y_i^2
\end{equation}


\subsection{Linear regression with statsmodels}
We can perform a linear regression in stats models using the following code

\begin{Verbatim}
model= sm.OLS(y,X) 
results = model.fit()
\end{Verbatim}
 This first command creates the ``model" object, which is all the information about the data and the statistical model (linear regression). The second commend actually does the computations which give the results.\footnote{The function OLS uses maximum likelihood as described in Chapter 8 of the textbook. This is constrast to the Bayesian methods used in Chapter's 6 and 8. We will learn how to approach this problem from a Bayesian perspective later on, but for now, note that the results are slightly different.}
 The results variable stores all the information about the fitted model, which we print with 
 \begin{Verbatim}
 print(results.summary())
 \end{Verbatim}
 
 This will output something like 
 \begin{Verbatim}
                            OLS Regression Results                            
==============================================================================
Dep. Variable:                      y   R-squared:                       0.132
Model:                            OLS   Adj. R-squared:                  0.123
Method:                 Least Squares   F-statistic:                     14.86
Date:                Mon, 13 Sep 2021   Prob (F-statistic):           0.000207
Time:                        21:28:39   Log-Likelihood:                 89.756
No. Observations:                 100   AIC:                            -175.5
Df Residuals:                      98   BIC:                            -170.3
Df Model:                           1                                         
Covariance Type:            nonrobust                                         
==============================================================================
                 coef    std err          t      P>|t|      [0.025      0.975]
------------------------------------------------------------------------------
const          1.9925      0.010    199.914      0.000       1.973       2.012
x1             0.1712      0.044      3.855      0.000       0.083       0.259
==============================================================================
Omnibus:                        2.592   Durbin-Watson:                   1.959
Prob(Omnibus):                  0.274   Jarque-Bera (JB):                2.382
Skew:                          -0.040   Prob(JB):                        0.304
Kurtosis:                       3.752   Cond. No.                         4.46
==============================================================================
\end{Verbatim}


\begin{example} Generate simulated data from the model
\begin{equation}
y \sim {\rm Normal}(4x + 5,5)
\end{equation} 
and perform a linear regression on the fitted model. 
\end{example}


\section{Correlation and standardization}

We start with an exercise in rewriting the estimator of $\beta$: 
Let $\hat{\sigma}_x^2$ and $\hat{\sigma}_y^2$ be the estimators of the variance in $x$ and $y$. Remember that even though we don't have a model for $x$, we can still compute the standard deviation. The sample correlation of two samples $x$ and $y$ is given by 
\begin{equation}
r_{x,y}^2 = \frac{1}{n-2}\sum (x_i - \bar{x})(y_i - \bar{y})
\end{equation}
Note that $\hat{\beta} = r_{x,y}/\hat{\sigma}_x^2$. Sometimes it is useful to \emph{standardize} the variables before performing a regression, meaning we transform them into standard Normal random variables. 
 \begin{exercise}  Show that for a sample of a random variable $x_1,x_2,\dots,x_n$
 \begin{equation}
 z_i = \frac{x_i - \bar{x}}{\hat{\sigma}_x^2}
 \end{equation}
\end{exercise}

If we perform a regression on the standardized variables, then 
\begin{equation}
\hat{\beta} 
\end{equation}

The quantity $\rho$ is called the \emph{correlation coefficient}. 
We will take 

\begin{exercise} Is the correlation coefficient symmetric; that is, i
\end{exercise}


\section{Hypothesis testing for regression models}
We've already discussed $p$-values and mentioned some potential reasons to avoid using them. However, they play a central role and statistics and we must therefore understand them in the context of regression. 

\section{Additional exercises} 
\begin{exercise} Using the code above, write a function which does the following: First, it generates data from a simulated linear regression with slope a, intercept b, measurment noise sigma and n data points. It then performs a linear regression on the simulated data and outputs the estimated slope, intercept, the $R^2$ value and the $p$-value for the slope.
You can generate the $x$ values by copying the code at the beginning of this section and changing $n$. 
\end{exercise}

\begin{exercise} Using your function, make a plot of the p-value vs. n using the values of a, b and sigma from above. Make the same plot with $a = 1.0$, $b = 2$ and $\sigma = 0.1$. Now repeat this but plotting $R^2$ instead of the $p$-values.  Can you explain why the behavoir of these plots makes sense? Hint: It may be helpful to plot the results on a log scale using `ax.semilogy`. 
\end{exercise}

\begin{exercise}
Generate data such that performing a linear regression results in a (a) A very small $p$-value and an $R^2$ value close to $1$. (b) A large $p$ value and a very small $R^2$ value. (c) A small $p$-value and a small $R^2$ value. 
\end{exercise}

\end{document}